% a mashup of hipstercv, friggeri and twenty cv
% https://www.latextemplates.com/template/twenty-seconds-resumecv
% https://www.latextemplates.com/template/friggeri-resume-cv

\documentclass[lighthipster]{simplehipstercv}
% available options are: darkhipster, lighthipster, pastel, allblack, grey, verylight, withoutsidebar
% withoutsidebar
\usepackage[utf8]{inputenc}
\usepackage[default]{raleway}
\usepackage[margin=1cm, a4paper]{geometry}


%------------------------------------------------------------------ Variablen

\newlength{\rightcolwidth}
\newlength{\leftcolwidth}
\setlength{\leftcolwidth}{0.23\textwidth}
\setlength{\rightcolwidth}{0.75\textwidth}

%------------------------------------------------------------------
\title{New Simple CV}
\author{\LaTeX{} Ninja}
\date{August 2023}

\pagestyle{empty}
\begin{document}


\thispagestyle{empty}
%-------------------------------------------------------------

\section*{Start}

\simpleheader{headercolour}{María Fernanda}{Jimenez Reyes}{Bióloga Marina. Doctora en Acuicultura}{white}



%------------------------------------------------

% this has to be here so the paracols starts..
\subsection*{}
\vspace{4em}

\setlength{\columnsep}{1.5cm}
\columnratio{0.23}[0.75]
\begin{paracol}{2}
\hbadness5000
%\backgroundcolor{c[1]}[rgb]{1,1,0.8} % cream yellow for column-1 %\backgroundcolor{g}[rgb]{0.8,1,1} % \backgroundcolor{l}[rgb]{0,0,0.7} % dark blue for left margin

\paracolbackgroundoptions

% 0.9,0.9,0.9 -- 0.8,0.8,0.8


\footnotesize
{\setasidefontcolour
\flushright
\begin{center}
    \roundpic{Mafe.jpg}
\end{center}

\bg{cvgreen}{white}{Formación}\\[0.5em]

{Bióloga marina de la Universidad Jorge Tadeo Lozano-Bogotá y Doctora en Acuicultura de la Pontificia Universidad Católica de Valparaíso}
\bigskip

\bg{cvgreen}{white}{personal} \\[0.5em]
María Fernanda Jiménez Reyes

nationalidad: Colombiana 

1984

\bigskip

\bg{cvgreen}{white}{Areas of especialización} \\[0.5em]

Ecología trófica ~•~ Microbiología ~•~ Probióticos ~•~ expresión génica ~•~ pesquería

\bigskip



\bigskip

\bg{cvgreen}{white}{Interests}\\[0.5em]

\lorem
\bigskip

\bg{cvgreen}{white}{Intereses}\\[0.5em]

\texttt{Ecología trófica} ~/~ \texttt{microbiología} ~/~ \texttt{expresión génica} ~/~ \texttt{descarte}



\vspace{4em}

\infobubble{\faAt}{cvgreen}{white}{mariafej.84@gmail.com}
\infobubble{\faFacebook}{cvgreen}{white}{Maria Fernanda Jimenez Reyes}
\infobubble{\faGithub}{cvgreen}{white}{MaferJi}

\phantom{turn the page}

\phantom{turn the page}
}
%-----------------------------------------------------------
\switchcolumn

\small
\section*{Resumen corto}

\begin{tabular}{r| p{0.5\textwidth} c}
    \cvevent{2008}{Bióloga marina}{Bogotá}{Colombia \color{cvred}}{Universidad de Bogotá Jorge Tadeo Lozano.\bigskip}{2023-08-29_08h39_02.png} \\
    \cvevent{2015}{Doctora en Acuicultura}{Valparaíso}{Chile \color{cvred}}{Pontificia universidad Católica de Valparaíso. \bigskip}{2023-08-29_08h40_02.png}
\end{tabular}
\vspace{3em}

\begin{minipage}[t]{0.35\textwidth}
\section*{Cursos}
\begin{tabular}{r p{0.6\textwidth} c}
    \cvdegree{2004}{Biología mamíferos marinos}{Biología e identificación}{Fundación Red Colombiana de Varamientos  \color{headerblue}}{}{disney.png} \\
    \cvdegree{2010}{Ecología parasitaria}{Bivalvos de interés comercial.}{Uruguay \color{headerblue}}{}{medal.jpeg} \\
    \cvdegree{2017}{Curso internacional}{Cambio Climático y Acuicultura.}{Chile \color{headerblue}}{}{medal.jpeg}
\end{tabular}
\end{minipage}\hfill
\begin{minipage}[t]{0.3\textwidth}
\section*{Programación}
\begin{tabular}{r @{\hspace{0.5em}}l}
     \bg{skilllabelcolour}{iconcolour}{Microsoft Office®} &  \barrule{0.4}{0.5em}{cvpurple}\\
     \bg{skilllabelcolour}{iconcolour}{REST 2009®} & \barrule{0.55}{0.5em}{cvgreen} \\
     \bg{skilllabelcolour}{iconcolour}{PhotoShop ®} & \barrule{0.5}{0.5em}{cvpurple} \\
     \bg{skilllabelcolour}{iconcolour}{RStudio} & \barrule{0.25}{0.5em}{cvpurple} \\
     \bg{skilllabelcolour}{iconcolour}{\LaTeX} & \barrule{0.1}{0.5em}{cvpurple} \\
\end{tabular}
\end{minipage}

\section*{Curriculum}
\begin{tabular}{r| p{0.5\textwidth} c}
    \cvevent{2018--Actualidad}{Investigador Semi-senior}{Valparaíso}{Chile \color{cvred}}{Proyecto de descarte pequeños pelágicos. \bigskip}{IFOP.png} \\
    \cvevent{2017--2018}{Investigador}{Valparaíso}{Chile \color{cvred}}{Seguimiento de pesquería de crustáceos demersales \bigskip}{IFOP.png} \\
\cvevent{2017}{Investigador asistente}{Santiago}{Chile \color{cvred}}{Virus respiratorios niños 0-2años e Influencia de ácidos grasos poliinsaturados en músculo esquelético. \bigskip}{UA.png} \\
    
\end{tabular}
\vspace{3em}

\begin{minipage}[t]{0.3\textwidth}
\section*{Premios \& Menciones honoríficas}
\begin{tabular}{>{\footnotesize\bfseries}r >{\footnotesize}p{0.55\textwidth}}
    2008-2015 & Beca interna PUCV \\
    2012 & Beca Conicyt \\
    
\end{tabular}
\bigskip

\section*{Idiomas}
\begin{tabular}{l | ll}
\textbf{Español} & C2 & {\phantom{x}\footnotesize Lengua materna} \\
\textbf{Inglés} & C2 & \pictofraction{\faCircle}{cvgreen}{3}{black!30}{1}{\tiny} \\
\textbf{Frances} & C2 & \pictofraction{\faCircle}{cvgreen}{1}{black!30}{3}{\tiny} \\
\end{tabular}
\bigskip

\end{minipage}\hfill
\begin{minipage}[t]{0.3\textwidth}
\section*{Publicaciones}
\begin{tabular}{>{\footnotesize\bfseries}r >{\footnotesize}p{0.7\textwidth}}
    2013 & \emph{Relaciones tróficas de cinco especies de peces de interés comercial en la bahía de Cartagena, Caribe colombiano}, Anales de Instituto de Investigaciones Marinas y Costeras. INVEMAR. \\
    2013 & \emph{Soybean meal induces intestinal inflammation in zebrafish larvae}, Plos One. \\
     2017 & \emph{Nota Científica: “Protocolo para obtención de alevines axénicos de trucha arcoíris (Oncorhynchus mykiss)”.}, LAJAR. \\
     2019 & \emph{Natural Compounds: A sustainable alternative to the phytopathogens control.},J. Chil. Chem. Soc. \\
    
\end{tabular}
\bigskip

\section*{Patentes}
\begin{tabular}{>{\footnotesize\bfseries}r >{\footnotesize}p{0.6\textwidth}}
    Nov. 2015 & ``Método para producir probióticos autóctonos con actividad inmunoestimulante y su uso en profilaxis contra flavobacteriosis en salmónidos'', at: \emph{WO2016037296 A1} INTA-Universidad de Chile, Nov. 2015.
\end{tabular}
\end{minipage}






\vfill{} % Whitespace before final footer

%----------------------------------------------------------------------------------------
%	FINAL FOOTER
%----------------------------------------------------------------------------------------
\setlength{\parindent}{0pt}
\begin{minipage}[t]{\rightcolwidth}
\begin{center}\fontfamily{\sfdefault}\selectfont \color{black!70}
{\small María Fernanda Jiménez Reyes \icon{\faEnvelopeO}{cvgreen}{} IFOP \icon{\faMapMarker}{cvgreen}{} Valparaíso-Chile \icon{\faPhone}{cvgreen}{} 56/9 3100 6316 \newline\icon{\faAt}{cvgreen}{} \protect\url{mariafej.84@gmail.com}
}
\end{center}
\end{minipage}

\end{paracol}

\end{document}
